

\usepackage{lscape}
\usepackage[english]{babel}

\usepackage{tabularx}
\usepackage{amsmath,amsfonts,amssymb,amsthm,epsfig,epstopdf,titling,url,array}
\usepackage[a4paper,margin=2cm,footskip=1cm]{geometry}
\usepackage{enumitem}
\usepackage{booktabs}
\usepackage{longtable}
\usepackage{listings}
\usepackage{fancyhdr}
\usepackage{fancyhdr}


\pagestyle{fancy}

\setlength{\columnsep}{1.5cm}
\setlength{\columnseprule}{0.2pt}

\fancypagestyle{IHA-fancy-style}{%
\fancyhf{} %Clear header and footer
%\fancyhead[LE,RO]{\slshape \rightmark} % section
%\fancyhead[LO,RE]{\slshape \leftmark} %chapter

\lhead{\makebox[\columnwidth]{ \leftmark }} %\slshape \leftmark \makebox[\columnwidth]{ \leftmark }
\rhead{\makebox[\columnwidth]{ \leftmark }} %\slshape \leftmark


\lfoot{\makebox[\columnwidth]{\thepage}}
\rfoot{\makebox[\columnwidth]{\number\numexpr\value{page}+1}\stepcounter{page}}

\renewcommand{\headrulewidth}{0.4pt} %Line at the header visible
\renewcommand{\footrulewidth}{0.0pt}} %Line at the footer visible


\pagestyle{IHA-fancy-style}


\newtheoremstyle{defstyle}
	% Abstand oben
	{5pt}
	% Abstand unten
	{5pt}
	% Schriftart
	{\normalfont} %original: \normalfont
	% Zeileneinzug (leer = kein Zeileneinzug, \parindent = Absatz-Zeileneinzug)
	{}
	% Titel Schriftart
	{\normalfont\bfseries}
	% Zeichen nach dem Titel
	{:} 
	% Leerraum nach dem Titel
	{ }
	% Lehrsatzkopf definieren (leer = normal)
	{#3}

\newtheoremstyle{thmstyle}
	% Abstand oben
	{10pt} %original: 10pt
	% Abstand unten
	{10pt} %original: 10pt
	% Schriftart
	{\normalfont} %original: \normalfont
	% Zeileneinzug (leer = kein Zeileneinzug, \parindent = Absatz-Zeileneinzug)
	{}
	% Titel Schriftart
	{\normalfont\bfseries}
	% Zeichen nach dem Titel
	{:} 
	% Leerraum nach dem Titel
	{ }
	% Lehrsatzkopf definieren (leer = normal)
	{\thmname{#1}\thmnumber{ #2}\thmnote{ (#3)}}

\theoremstyle{defstyle}
% Definition
\newtheorem*{definition}{Definition}

\theoremstyle{thmstyle}
% 
\newtheorem*{theorem}{Theorem}
% Beachte
\newtheorem*{tnote}{Note}

% Beispiel
\newtheorem*{example}{Example}

\newenvironment{note}
	{\begin{snugshade}\begin{tnote}}
	{\end{tnote}\end{snugshade}}
	
\newenvironment{highlight}
	{\begin{snugshade}}
	{\end{snugshade}}
	
	
\newcommand{\R}{\mathbb{R}}
\newcommand{\N}{\mathbb{N}}
\renewcommand{\P}{\mathbb{P}}
\newcommand{\E}{\mathbb{E}}
