

% %NOTE:
% % CHECKED WITH SLIDES: YES!
% % CHECKED WITH EXERCISES: NO -- TODO
% % MISSING: Nothing important ;)


\section{Real Time Models}

\subsection{Classification}
\begin{definition} [Preemptive algorithm]
With preemptive algorithms, the running task can be 
interrupted at any time to assign the processor to another 
active task, according to a predefined scheduling policy.
\end{definition}

\begin{definition} [Non-preemptive algorithm]
With a non-preemptive algorithm, a task, once started, is 
executed by the processor until completion.
\end{definition}

\begin{definition} [Static Algorithm]
Static algorithms
are those in which scheduling decisions 
are based on fixed parameters, assigned to tasks before 
their activation.
\end{definition}

\begin{definition}[Dynamic algorithms]
Dynamic algorithms
are those in which scheduling 
decisions are based on dynamic parameters that may 
change during system execution

\end{definition}

\begin{definition}[Optimal]
An algorithm is said optimal if it minimizes some given 
cost function defined over the task set.
\end{definition}

\begin{definition}[Heuristic]
An algorithm is said to be heuristic if it tends toward but 
does not guarantee to find the optimal schedule.
\end{definition}

\subsection{General}

\begin{definition}[Hard]
A real-time task is said to be hard, if missing its 
deadline may cause catastrophic consequences on the 
environment under control.
\end{definition}

\begin{definition}[Soft]
A real-time task is called soft, if meeting its deadline is 
desirable for performance reasons, but missing its deadline 
does not cause serious damage to the environment and does 
not jeopardize correct system behavior.
\end{definition}



\subsection{Schedule and Timing}


\begin{definition}[Schedule]
A schedule is an assignment of tasks to the processor, such 
that each task is executed until completion. It can be defined as an integer step function
where $ \sigma : \R \rightarrow \N $ denotes the task which is executed at time $t$. 
\end{definition}

\begin{definition}[Idle]
If $ \sigma (t) = 0$ then the processor is called idle. So no task is executed.
\end{definition}

\begin{definition}[Context switch]
If $\sigma (t) $ changes its value at some time.
\end{definition}

\begin{definition}[Time slice]
Interval, in which $\sigma (t) $ is constant.
\end{definition}

\begin{definition}[Feasible]
A schedule is said to be feasible, if all task can be completed 
according to a set of specified constraints.
\end{definition}

\begin{definition}[schedulable]
A set of tasks is said to be schedulable, if there exists at 
least one algorithm that can produce a feasible schedule.
\end{definition}


\subsection{Metrics}

\begin{definition}[Arrival time $a_i$/release time $r_i$]
Is the time at which a task becomes ready for execution (Waits in ready queue).
\end{definition}

\begin{definition}[Computation time $C_i$]
Is the time necessary to the processor for executing the task without interruption.
\end{definition}

\begin{definition}[Deadline $d_i$]
Is the absolute time at which a task should be completed.
\end{definition}

\begin{definition}[Fair Schedule]
(for the first exercise) a schedule is fair, if every task eventually gets a chance to execute on the processor.
\end{definition}

\begin{definition}[Finishing time $f_i$]
Is the time at which a task finishes its execution.
\end{definition}

\begin{definition}[Lateness $L_i$]
$L_i = f_i - d_i$, represents the delay of a task 
completion with respect to its deadline.
\end{definition}

\begin{definition}[Laxity / slack time $ X_i$]
$X_i = d_i - a_i -C_i$, is the maximum time a task can be delayed on its activation to complete 
within its deadline.
\end{definition}

\begin{definition}[Response time]
The amount of time it takes to finish executing a task: $response_i = f_i - a_i$.
\end{definition}

\begin{definition}[Start time $s_i$]
Is the time at which a task starts its execution.
\end{definition}

\begin{definition} [Tardiness / exceeding time $ E_i$]
$E_i = max(0, L_i)$, is the time a task stays active after its deadline.
\end{definition}

\begin{definition}[Throughput]
The measure of work done in a unit time interval.
\end{definition}

\begin{definition}[Utilization]
Ratio of busy time of the processor to the total time required for all tasks to finish.
\end{definition}

\begin{definition}[Waiting time]
Time spent by a task in the ready queue.
\end{definition}


\begin{definition}[Average response time]
\begin{equation} 
t_r = \frac{1}{n} \sum\limits_{i = 1}^{n}(f_i - r_i) 
\end{equation}
\end{definition}

\begin{definition}[Total completition time]
\begin{equation}
t_c = \max_i (f_i) - \min_i ( r_i)
\end{equation}
\end{definition}

\begin{definition}[Weighted sum of response time]
\begin{equation}
t_w = \frac{ \sum\limits_{i = 1}^{n}w_i(f_i - r_i) }{\sum\limits_{i = 1}^{n} w_i}
\end{equation}
\end{definition}

\begin{definition} [Maximum lateness]
\begin{equation}
L_{max} = \max_i(f_i - d_i)
\end{equation}
\end{definition}

\begin{definition}[Number of late tasks]

\begin{equation}
 { N_{late} = \sum\limits_{i = 1}^{n} miss(f_i)  }	\qquad
 { miss(f_i) = \left\{
	\begin{array}{ll}
		0  & \mbox{if } f_i \leq d_i \\
		1 & \mbox{otherwise }
	\end{array}
\right.
}
\end{equation}

\end{definition}



\cleardoublepage