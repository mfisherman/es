
% %NOTE:
% % CHECKED WITH SLIDES: YES!
% % CHECKED WITH EXERCISES: NO -- TODO
% % MISSING: -

\section{System Components}
\subsection{General Purpose Processors}


\subsubsection{Properties}
\begin{itemize}[noitemsep]
\item High Performance: Use of parallelism, optimized circuits and technology
\item Complex memory architecture
\item Not suited for real-time applications: times are highly unpredictable due to intensive resource sharing and dynamic decisions
\item Good average performance
\item High power consumption
\end{itemize}

\subsubsection{Multicore Processors}
Properties
\begin{itemize}[noitemsep]
\item Higher execution performance by parallelism
\item useful in high-performance embedded systems (autonomous driving)
\item Disadvantages and problems for embedded systems:
	\begin{itemize}[noitemsep]
	\item Increased interference on shared resources (buses, shared caches)
	\item Increased timing uncertainty
	\item Often no parallelism in embedded applications
	\end{itemize}
\end{itemize}

Examples
\begin{itemize}[noitemsep]
\item Xeon Phi
\item Oracle Sparc
\end{itemize}

Domains
\begin{itemize}[noitemsep]
\item Image and Audio processing
\item Signal Processing
\item Scientific computing
\item Control
\end{itemize}

\subsection{System Specialization}
Main difference between general purpose highest 
volume microprocessors and embedded systems is 
specialization.
\subsubsection{Flexibility}
\begin{itemize}[noitemsep]
\item application domain specific systems shall cover a class of 
applications
\item some flexibility is required to account for late changes, 
debugging
\end{itemize}

\subsubsection{Examples}
\begin{itemize}[noitemsep]
\item Code-size efficiency: RISC designed for run-time efficiency. Many instructions for simple job
\item Heterogeneous registers: Different functionalities for each register vs. general purpose register
\item Specified memory banks
\end{itemize}


\subsubsection{Microcontroller}
\begin{itemize}[noitemsep]
\item for control-dominant applications
\item Short latency times
\item Low power consumption
\item Suited for real-time
applications 
\end{itemize}

\subsection{Digital Signal Processors/VLIW}
\subsubsection{General}
\begin{itemize}[noitemsep]
\item Streaming oriented systems
with mostly periodic 
behavior 
\item Application examples: signal processing, control 
engineering
\item use of parallel hardware units (VLIW)
\item high data throughput
\item specialized memory
\item suited for real-time
applications
\end{itemize}

\subsubsection{Very Long Instruction Word - VLIW}
Key idea: detection of possible parallelism to be done by compiler, not by hardware at run-time (inefficient). VLIW: parallel operations (instructions) encoded in one long word (instruction packet), each instruction controlling one functional unit.


\subsection{Programmable Hardware - FPGA}
\begin{itemize}[noitemsep]
\item Granularity of logic units: Gate, tables, memory, functional blocks (ALU, control, data 
path, processor)
\item Communication network: Crossbar, hierarchical mesh, tree
\item Reconfiguration: fixed at production time, once at design time, dynamic during run-time
\end{itemize}

\subsection{Application Specific Circuits - ASICs}
\begin{itemize}[noitemsep]
\item Custom-designed circuits necessary if ultimate speed or energy efficiency is the goal and large numbers can be sold
\item Problems: long design times, lack of flexibility (changing standards) and high costs
\end{itemize}

\subsection{System on a Chip}
A system on a chip or system on chip (SoC or SOC) is an integrated circuit (IC) that integrates all sniper components of a computer or other electronic system into a single chip. It may contain digital, analog, mixed-signal, and often radio-frequency functions—all on a single chip substrate. SoCs are very common in the mobile electronics market because of their low power consumption. A typical application is in the area of embedded systems.

\cleardoublepage